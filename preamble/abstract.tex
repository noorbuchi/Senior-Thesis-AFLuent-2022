\unnumberedchapter{Abstract}
\chapter*{Abstract}

Debugging code is one of the most time consuming and costly tasks for
developers. To reduce the time spent on debugging code, many efforts have been
focused to automate this process. Spectrum-Based fault localization is one of
the ways developers can rank statement based on their potential of being faulty.
This research proposes \href{https://github.com/noorbuchi/AFLuent}{\emph{AFLuent}},an easy to use Spectrum-Based
fault localization tool integrated directly into Python's testing framework,
Pytest, making it user friendly for installation and usage. The
tool implements four main SBFL equations, Tarantula, Ochiai, Ochiai2, and DStar,
which are evaluated in this research. Additionally, several fixes are
proposed and evaluated in aims to reduce the frequency of ties when ranking
statements. By parsing the syntax tree of the code, several indicators can be
analyzed to determine error prone statements and create a metric that breaks
the ties produced by the SBFL equations. While the results of this research do
not declare a clear winner, a new understanding of the effectiveness of
tie breaking approaches is achieved. The study concludes that the logical
variants of Ochiai, Ochiai2, and DStar, while very similar to each other,
outperformed the rest of the approaches. Lastly, the study evaluates the time
overhead introduced by AFLuent and the time cost developers must pay if they choose
to use the tool. Integrating AFLuent into small scale projects brings very
little increase in the time taken to run the test, however, the performance
becomes worse as the codebase increases in size. There are many possibilities of extending
this work, improving effectiveness, efficiency, and performing
a more thorough evaluation can significantly increase confidence in AFLuent's
ability to locate faults.